\documentclass[11pt]{article}

\usepackage{listings}
 \lstset{
   basicstyle=\scriptsize\ttfamily,
   keywordstyle=\bfseries\ttfamily\color{orange},
   stringstyle=\color{green}\ttfamily,
   commentstyle=\color{middlegray}\ttfamily,
   emph={square}, 
   emphstyle=\color{blue}\texttt,
   emph={[2]root,base},
   emphstyle={[2]\color{yac}\texttt},
   showstringspaces=false,
   flexiblecolumns=true,
   tabsize=2,
   numbers=left,
   numberstyle=\tiny,
   numberblanklines=false,
   stepnumber=1,
   numbersep=10pt,
   xleftmargin=1 pt
 }
\begin{document}

\section{LaufzeitAnalysator}
In dieser Methode wird die Laufzeit der einzelnen Gatter abhängig von äu"seren Faktoren berechnet
\begin{lstlisting}
	void LaufzeitAnalysator::berechne_LaufzeitEinzelgatter() /// berechnet Laufzeit fuer jedes einzelne Gatter
{
    double spgFaktor,
    tmpFaktor,
    przFaktor;

    faktoren->getFaktoren(spgFaktor, tmpFaktor, przFaktor); /// holt sich aeussere Faktoren ueber Referenz

    debug_msg( "INFO: SPG-F: "<<spgFaktor<<" TMP-F: "<<tmpFaktor<<" PRZ-F: "<<przFaktor<<endl<<endl);

    for (ListenElement* ptr = gE->getStartElement(); ptr != NULL; ptr = ptr->getNextElement()) /// durchlaeuft die ListenElemente und weist jedem Schaltwerk im Listenelement seine
    {

        double tpd0 = ptr->getSchaltwerkElement()->getTyp()->getGrundLaufzeit(); /// Grundlaufzeit tpd0 (in ps) und seinen
        double lastFaktor = ptr->getSchaltwerkElement()->getTyp()->getLastFaktor(); /// Lastfaktor lastFaktor (in fs/fF) zu
        double last_C = 0;

        for (int i = 0; i < (ptr->getSchaltwerkElement()->getAnzahlNachfolger()); i++ ) /// summiert die Eingangs-C (in fF) der mit dem Ausgang verbundenen Gatter und
        {

            last_C = last_C + ptr->getSchaltwerkElement()->getNachfolger(i)->getTyp()->getLastKapazitaet(); /// speichert diese im Schaltwerkelement
            debug_msg("INFO: C-Last: \t"<<last_C<<endl);

        }
        /// t_pd,actual = (t_pd0 + LastF + LastC) * TempF * SpgF * PrzF
        // (ps) = ( ps + (fs/fF) * fF * 1000) //wieso funkionierts mit 1/1000 und nicht mit 1000 ???????
        ptr->getSchaltwerkElement()->setLaufzeitEinzelgatter(((tpd0 + lastFaktor * last_C * 0.001) * spgFaktor * tmpFaktor * przFaktor)); /// berechnet die Gesamtlaufzeit des Einzelgatters

        debug_msg("INFO: Laufzeit Einzelgatter von "<< ptr->getSchaltwerkElement()->getName() << ":\t"<<ptr->getSchaltwerkElement()->getLaufzeitEinzelgatter()<<endl);
    }

}

\end{lstlisting}

Die Zyklensuche pr"uft die Schaltung auf einen eventuell vorhandenen Zyklus, damit die Laufzeitanalyse darauf reagieren kann
\begin{lstlisting}
	bool LaufzeitAnalysator::zyklensuche(SchaltwerkElement* se)
{
    for(SchaltwerkElement* i=se ; i!=NULL ; i=DFS_Zwischenspeicher[i].VaterElement)
    {
        if( DFS_Zwischenspeicher[i].VaterElement == se )
        {
            return true;
        }
    }
    return false;
}


\end{lstlisting}


Die Tiefensuche durchsucht die einzelnen "Aste des Baumes nach evtuellen Zyklen und berechnet den l"angsten "Ubergangspfad sowie Ausgangspfad

\begin{lstlisting}
void LaufzeitAnalysator::DFS_Visit(SchaltwerkElement* k,SchaltwerkElement* s)
{
    for (int i=0; i<k->getAnzahlNachfolger(); i++)
    {
        if(zyklusFound) {
            break;
            debug_msg( "Zyklus gefunde, breche ab!" << endl);
        }


        SchaltwerkElement* v =k->getNachfolger(i);
        debug_msg("nachfolger:"<<v->getName()<<endl);

        if (v->getTyp()->getIsFlipflop())
        {
            debug_msg("ff gefunden: "<<v->getName()<<endl<<endl<<endl);

            if (laufzeitUebergangspfad<DFS_Zwischenspeicher[k].PfadLaufzeit + k->getLaufzeitEinzelgatter())
            {
                laufzeitUebergangspfad=DFS_Zwischenspeicher[k].PfadLaufzeit + k->getLaufzeitEinzelgatter();

                uebergangspfad = k->getName() + " ->" + k->getNachfolger(i)->getName();
                //cout << "Übergangspfad: "<<uebergangspfad<<":"<<laufzeitUebergangspfad<<endl;
                //String erstellen
                for( SchaltwerkElement* v = k ; v != s ; v = DFS_Zwischenspeicher[v].VaterElement )
                {
                    uebergangspfad.insert( 0 , ( DFS_Zwischenspeicher[v].VaterElement->getName() + "-> " ));
                }

            }
        }

        else if (DFS_Zwischenspeicher[v].PfadLaufzeit < (DFS_Zwischenspeicher[k].PfadLaufzeit +k->getLaufzeitEinzelgatter()))
        {

            if( ( ( DFS_Zwischenspeicher[ v ].PfadLaufzeit != 0 ) or ( v== s ) ) and ( DFS_Zwischenspeicher[ v ].VaterElement != k) )
            {

                DFS_Zwischenspeicher[v].VaterElement =k;


                if(zyklensuche(v))
                {
                    zyklusFound = true;
                    cout << "Fehler Zyklensuche!"<<endl;

                }
            }
            DFS_Zwischenspeicher[v].PfadLaufzeit = DFS_Zwischenspeicher[k].PfadLaufzeit + k->getLaufzeitEinzelgatter();
            DFS_Zwischenspeicher[v].VaterElement = k;

            DFS_Visit(v,s);

        }

    }

    if(k->getIsAusgangsElement() and (laufzeitAusgangspfad < (DFS_Zwischenspeicher[k].PfadLaufzeit + k->getLaufzeitEinzelgatter())))
    {
        laufzeitAusgangspfad = DFS_Zwischenspeicher[k].PfadLaufzeit + k->getLaufzeitEinzelgatter();

        ausgangspfad = k->getName();

        for(SchaltwerkElement * j = k; j != s; j= DFS_Zwischenspeicher[j].VaterElement)
        {
            ausgangspfad.insert(0,(DFS_Zwischenspeicher[j].VaterElement->getName() + "->"));
        }

    }

}
\end{lstlisting}



Im Folgenden wird die maximale Frequenz berechnet, mit welcher die Schaltung betrieben werden kann und gepr"uft ob die vorgesehene Frequenz die maximale Frequenz nicht "uberschreitet

\begin{lstlisting}
double LaufzeitAnalysator::maxFrequenz(long freq)
{

    cout << "Längster Pfad im Überfuehrungsschaltnetz:" << endl;
    cout << uebergangspfad << endl << endl;
    cout << "Maximale Laufzeit der Pfade im Überfuehrungsschaltnetz: " << laufzeitUebergangspfad << " ps" << endl;
    cout << endl;
    cout << "Längster Pfad im Ausgangsschaltnetz:" << endl;
    cout << ausgangspfad << endl << endl;
    cout << "Maximale Laufzeit der Pfade im Ausgangsschaltnetz: " << laufzeitAusgangspfad << " ps" << endl;
    cout << endl;
    cout << "-----------------------------------------------"<<endl;
    long double maxF = 1/ (dynamic_cast<Flipflop*>(( gE->getBibliothek()->getBibElement("dff")))->getSetupTime() * 0.000000000001 + laufzeitUebergangspfad * 0.000000000001 );

    if(maxF>1000){
        if(maxF>1000000){

        cout << "maxFrequenz: "<<maxF/1000000<<" MHz"<<endl;
        }
        else{
            cout << "maxFrequenz: "<<maxF/1000<<" kHz"<<endl;
        }
    }

    cout << "-----------------------------------------------"<<endl;
    if(maxF < freq){
        cout << "Frequenz zu groß"<<endl;
    }
    else{
        cout << "Frequenz okay"<<endl;

    }


}
\end{lstlisting}

\begin{lstlisting}
\end{lstlisting}

\end{document}
